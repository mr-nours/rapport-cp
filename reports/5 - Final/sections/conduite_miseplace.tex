\section{Déroulement du projet}

\subsection{Réunions de projet}

Le projet a été rythmé par une succession de 3 sprints comprenant des réunions de projet.
Deux réunions étaient programmées chaque semaine, une avec le client et une entre les membres de l'équipe de développement.
La première réunion avec le client avait pour but de fixer les différents objectifs et de présenter les avancées de la semaine. Cette réunion était utile, car elle nous a permis de poser directement nos questions au client afin de mettre immédiatement au clair les zones floues du projet, et lui pouvait faire des remarques sur l'avancement en contrepartie. Les réunions hebdomadaires avec l'équipe ont quant à elles permis d'avoir un suivi rigoureux et quantifiable de la gestion du projet. En effet, lors de chaque réunion, plusieurs points prédéfinis étaient systématiquement discutés.  
Tout d'abord, la planification des sprints. Suite à la réunion avec le client, nous faisions une planification rigoureuse pour la semaine à venir. Nous discutions donc de nos objectifs à atteindre pour la semaine suivante, ce qui impliquait alors une mise à jour de notre tableau de bord \emph{Youkan}. Les idées d'implémentations étaient également discutées lors de cette réunion, de même que la manière de s'y prendre pour résoudre tel problème et comment le travail allait être réparti, répartition qui se faisait toujours en fonction des capacités, des préférences et des compétences de chacun. 

\subsection{Gestion des risques}

Afin d'avoir une meilleure prise sur le projet, nous avons pris l'habitude d'anticiper les risques potentiels (par exemple l'impossibilité d'utiliser telle ou telle bibliothèque, un problème de mise en place de l'environnement \emph{android}, une mauvaise conception ou un problème d'intégration). Pour chacun de ces risques, nous avons alors anticipé une solution de contournement rapide. Cette bonne pratique nous a permis de rapidement rebondir en cas de problème concret.    

\subsection{Mêlée quotidienne}

Pour nous tenir informés de l'avancement du projet les uns les autres et afin de s'entraider, nous avons pris l'habitude d'échanger régulièrement par mail. Nous avons donc créé un répertoire dédié à la conduite de projet sur nos messageries respectives en créant un filtre sur les messages comportant dans le sujet la mention [PCP].
Ainsi, un message étant toujours envoyé à l'ensemble du groupe, chacun pouvait contribuer à l'entraide et pouvait se tenir informé en temps réel des avancées et des problèmes rencontrés. De plus, des entrevues rapides avec l'équipe pouvaient avoir lieu quotidiennement si besoin.

\subsection{Utilisation de la burndown Chart}

Explication de la burndownChart

\subsection{Réunion de rétrospection}

À chaque fin de sprint, une réunion de rétrospection a eu lieu pour discuter des points relatifs à la gestion du projet ainsi que de l'application de la méthode \emph{Scrum} en vue de son amélioration continue. Lors de cette réunion, chacun a fait le compte rendu de son travail en expliquant les objectifs qu'il a atteints et les difficultés qu'il a rencontrées. Cette réunion nous a alors permis de faire ressortir les points positifs et les points négatifs rencontrés durant le sprint, des points à améliorer ou à revoir. Ainsi donc, nous avons pu améliorer de façon itérative notre processus \emph{Scrum} et notre façon de travailler et de gérer nos risques en nous servant de chaque débriefing de sprint.  