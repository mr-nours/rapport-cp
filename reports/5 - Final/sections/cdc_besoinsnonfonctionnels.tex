\section{Besoins non-fonctionnels}

\subsection{Refactoring}

Le refactoring consiste à retravailler un code source dans le but d'améliorer sa lisibilité et son efficacité, et de simplifier sa maintenance.
L'introduction de nouvelles fonctionnalités induit le besoin de refactoriser souvent le code afin de simplifier
la maintenance et la compréhension.
En plus d'un simple nettoyage, cela nous amène à vérifier que notre architecture répond toujours aux
objectifs fixés.

L'objectif est bien sûr d'obtenir un gain de clarté, de lisibilité, de maintenabilité, et probablement de performances. Nous pouvons ainsi continuer l'ajout de fonctions sur une base saine.

\subsection{Ergonomie}

Le besoin d'ergonomie se fait ressentir car les utilisateurs visés ne sont pas spécialement adeptes des dernières technologies informatiques et ne disposent pas du temps nécessaire au suivi d'une formation préalable à l'utilisation de l'application. L'application doit donc être au possible intuitive et facile à utiliser, malgré les contraintes matérielles. Pour un terminal dont la zone d'affichage est réduite, nous visons donc à maximiser la taille des contrôles tout en minimisant leur interférence sur les zones d'affichage. Cela traduit aussi le besoin de rendre rapide l'accès aux informations des examens.

\subsection{Performance}

L'utilisabilité de l'application passe par ses performances. La puissance de calcul étant limitée sur les terminaux portables, l'enjeu principal est donc de limiter l'usage des ressources tout en maximisant la fluidité de l'interface graphique.
