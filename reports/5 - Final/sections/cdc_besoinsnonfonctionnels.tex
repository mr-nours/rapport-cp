\section{Besoins non-fonctionnels}

\subsection{exemple (refactoring}\label{refactoring} 

Le refactoring consiste à retravailler un code source dans le but d'améliorer sa lisibilité et son efficacité, et de simplifier sa maintenance.
L'introduction de nouveaux patterns induit le besoin de refactoriser souvent le code afin de simplifier
le codage et la compréhension.
En plus d'un simple nettoyage, cela nous amène à vérifier que notre architecture répond toujours aux
objectifs fixés.

L'objectif est bien sûr d'obtenir un gain de clarté, de lisibilité, de maintenabilité, et probablement de performances. Nous pouvons ainsi continuer l'ajout de fonctions sur une base saine.