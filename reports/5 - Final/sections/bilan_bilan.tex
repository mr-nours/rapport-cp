\section{Bilan}

En conclusion, nous pouvons dire que le bilan général est plutôt bon. Nous avons globalement satisfait les besoins client tout en réalisant nos objectifs initiaux en terme de besoins non fonctionnels.

Au niveau de l'architecture, les méthodes sont restées simples, ce qui limite selon nous les sources de bogues. Nous limitons les dépendances avec les bibliothèques \emph{android}. L'application est donc plus maintenable et évolutive.
Cela s'inscrit dans la logique des méthodes agiles, que nous avons adoptées et éprouvées. 

Au niveau de la gestion du projet, bien que le déploiement et l'utilisation des outils furent difficiles de premier abord, nous nous sommes rapidement insérés dans des logiques agiles et nous avons pu mesurer les bénéfices qu'apportait la méthode \emph{scrum}. Nous retiendrons les aspects communication et itératif, qui constituent la partie la plus importante dans l'application des méthodes agiles.

\subsection{Échecs}

Malgré ce bilan positif, nous déplorons le manque de disponibilité de chaque membre de l'équipe de développement, due au contexte académique, qui nuit à l'efficacité des méthodes agiles. En effet, ces méthodes demandent un investissement important pour l'aspect communication, et sans cet aspect-là, l'encadrement des ressources du projet devient plus difficile, et la qualité du pilotage du projet s'amoindrit. Nous souhaitons appliquer la méthode \emph{scrum} dans un contexte plus favorable pour pouvoir nous perfectionner et nous préparer à son utilisation dans le cadre professionnel.

