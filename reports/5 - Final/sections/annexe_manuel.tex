\section{Installation}

\section{Ouvrir un examen}

Au démarrage de l'application, vous arrivez sur l'écran d'exploration des fichiers. Avec l'explorateur, vous pouvez visualiser les dossiers contenus dans la carte SD de votre appareil. En cliquant sur un dossier quelconque, vous en visualisez le contenu. Pour revenir en arrière, appuyez sur le dossier "..". Pour ouvrir un examen, déplacez vous dans le dossier qui le contient, et cliquez sur le nom du patient qui s'affiche alors dans la liste (exemple en figure \vref{fichiers}).

Lorsque vous ouvrez un examen, la fenêtre d'interaction s'ouvre (\vref{exam-interaction} pour le manuel relatif à cette fenêtre).

\begin{note} %Vous pouvez utiliser 'question' et 'attention' aussi. Le 'attention' pique les yeux par contre...
Les examens \emph{DICOM} sont stockés dans des dossiers dont le nom peut être très complexe. Pour simplifier la reconnaissance des différents examens, l'application renomme les dossiers qui contiennent les fichiers \emph{DICOM} afin qu'ils identifient les patients correspondants.
\end{note}

\section{Interagir avec l'examen} \label{exam-interaction}

Après ouverture d'un examen, la première coupe est affichée automatiquement. La navigation entre les coupes de l'examen est possible en utilisant les flèches de défilement. Le menu disponible en appuyant sur la touche menu de l'appareil propose de sauvegarder l'examen ouvert au format bmi3d et de visualiser les informations relatives à l'examen et à la coupe actuelle.

\subsection{Contrôle de la coupe}
La partie centrale de la fenêtre affiche la coupe actuelle. On peut effectuer un zoom en utilisant le curseur au dessus de la vue. On peut déplacer la coupe grâce à l'écran tactile. Le zoom et la position actuelle sont conservés lors des changements de coupe.

\subsection{Contrôle du contraste}
Le bouton "Hounsfield controls" situé sous la vue de la coupe permet de dérouler les contrôles de contraste. Le contraste utilise l'échelle de Hounsfield. Le curseur "Center" permet de régler la position de la fenêtre de valeurs et le curseur "Width", la taille de cette fenêtre. Des présélections de valeurs courantes peuvent être appliquées via le menu déroulant situé sous les curseurs.

\subsection{Mode dessin}
Le bouton situé en haut à gauche de l'écran permet de basculer entre le mode déplacement et le mode dessin. Ce mode donne la possibilité de sélectionner des zones par un dessin sur l'écran tactile. Deux outils sont disponibles : le crayon pour tracer les sélections et la gomme pour corriger les tracés. Le changement d'outil et le réglage de l'épaisseur du tracé se font à l'aide des boutons situés en haut à droite de la fenêtre.