\section{Installation}

\section{Ouvrir un examen}

Au démarrage de l'application, vous arrivez sur l'écran d'exploration des fichiers. Avec l'explorateur, vous pouvez visualiser les dossiers contenus dans la carte SD de votre appareil. En cliquant sur un dossier quelconque, vous en visualisez le contenu. Pour revenir en arrière, appuyez sur le dossier "..". Pour ouvrir un examen, déplacez vous dans le dossier qui le contient, et cliquez sur le nom du patient qui s'affiche alors dans la liste (exemple en figure \vref{fichiers}).

Lorsque vous ouvrez un examen, la fenêtre d'interaction s'ouvre (\vref{exam-interaction} pour le manuel relatif à cette fenêtre).

\begin{note} %Vous pouvez utiliser 'question' et 'attention' aussi. Le 'attention' pique les yeux par contre...
Les examens \emph{DICOM} sont stockés dans des dossiers dont le nom peut être très complexe. Pour simplifier la reconnaissance des différents examens, l'application renomme les dossiers qui contiennent les fichiers \emph{DICOM} afin qu'ils identifient les patients correspondants.
\end{note}

\section{Interagir avec l'examen} \label{exam-interaction}