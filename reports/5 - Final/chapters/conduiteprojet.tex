\chapter{Gestion de projet}
\minitoc

\section{La Gestion du projet}
%Explication et distribution des roles scrum, vision du projet (objectifs, jalons, utilisateurs visés) et mise en place du backlog, estimations des points.

\subsection{Méthodes et outils Scrum}

Le projet s'organise autour de la méthode agile \emph{Scrum}, méthode qui nous a permis de maîtriser notre production, de la quantifier et de la planifier. Ce projet a été réalisé en groupe avec une durée fixe à respecter et des objectifs clairs exprimés en début de projet (confère cahier des charges).

Afin de piloter au mieux le projet, nous avons utilisé l'outil \emph{Youkan}, qui permet de gérer tout le processus projet. Cet outil permet d'avoir une bonne traçabilité favorisant ainsi la collaboration entre les différents acteurs du projet.
Nous avons utilisé \emph{Youkan} pour planifier et suivre les différentes étapes de notre projet et modéliser les exigences du client en intégrant toutes demandes de modifications.

Nous avons employé la méthode \emph{Scrum} pour établir les estimations sur la difficulté et la durée des différentes tâches à réaliser. Pour cela, nous avons mis en place un \emph{backlog}, où nous avons listé les différentes tâches que nous avions à effectuer, et pour chacune d'entre elles, nous avons tous pondéré la tâche en fonction de sa difficulté à l'aide d'un système de points et de moyennes. 

\subsection{Découpage du projet}

Le projet a été découpé en 3 sprints, un sprint équivalent à une période de 21 jours. C'est donc un processus itératif qui a été utilisé tout au long de la réalisation du projet. Le premier sprint consistait en l'évaluation des valeurs réelles de nos pondérations, c'est à dire à combien de jours réels correspondait une pondération. Ainsi, nous avons pu obtenir des estimations précises du temps nécessaire pour réaliser une tâche en fonction des compétences et de la cadence de travail de chacun des membres du groupe. 
De plus, \emph{Youkan} dispose d'un Task Board faisant office de tableau virtuel sur lequel nous pouvons coller des post-its, post-it que nous pouvons classer dans une des quatre rubriques suivantes : Todo, In progress, Test et Done. L'ensemble des acteurs du projet peut se servir de ce tableau pour suivre avec précision l'avancée du projet.

\subsection{Organisation de l'équipe}

Afin de piloter l'équipe, un responsable de gestion nommé \emph{Scrum Master} a été désigné. Son rôle est de s'assurer du bon déroulement du projet ainsi que de l'application de la méthode \emph{Scrum}. Il a donc ainsi veillé à bien communiquer la vision et les objectifs du projet entre les acteurs de celui-ci. C'est lui qui était chargé d'échanger avec le client et de restituer, lors de réunions organisées, les différentes attentes du client ainsi que les différents objectifs à atteindre durant le sprint. Pour faciliter les échanges, le Scrum Master a fixé un jour dans la semaine afin de pouvoir échanger et faire le point avec l'équipe de manière hebdomadaire. Le Scrum Master a également accompagné l'équipe en demandant un retour des différents développeurs lors des entrevues quotidiennes, en répartissant le travail selon les compétences et préférences de chacun et en envoyant des mails de rappel ou de relance lorsque cela semblait nécessaire. Le Scrum Master a veillé au maintien d'une bonne cohésion de groupe, un point important du projet.

Enfin, nous avons été tout au long du projet critiques vis-à-vis de l'application de notre méthode de travail en essayant d'évaluer rétrospectivement et de manière objective, lors d'une réunion dédiée, les processus du projet (ce que nous avons du mal à appliquer, ce que nous devons améliorer, une estimation mauvaise à revoir, etc.). 

\subsection{Gestion des risques}

A l'aide de \emph{Youkan} et de la méthode \emph{Scrum}, nous avons pu identifier et suivre nos risques plus efficacement. Nous avons pris l'habitude de mettre en place des solutions de contournements rapides afin d'anticiper les facteurs à risque pouvant ralentir la progression du projet tout en restant flexibles.
Les paramètres de planifications ont également été surveillés de prêt grâce à la Burndown chart, qui représente graphiquement la charge de travail restante prévisionnelle et la charge de travail restante concrète.

\subsection{Mise en place des jalons}

Afin de garantir un projet fonctionnel, nous avons placé des jalons à chaque fin de sprint dans le but de fournir un livrable. Cette exigence nous a obligés a garantir un fonctionnement du produit tout au long du projet en déployant le produit réalisé sur tablette avant chaque livraison. Cela a également permis au client de voir l'avancée du projet et de pouvoir émettre immédiatement des remarques, des demandes ou des critiques concernant les livrables, nous permettant ainsi d'appliquer rapidement les correctifs nécessaires et d'aboutir à un produit fini correspondant au mieux aux attentes du client. 

\subsection{Pré-requis du projet}
%Choix des outils de dév. (codage, versionnage, bug tracking, integration et tests). Choix des techniques d'integration et des conventions de codage.
